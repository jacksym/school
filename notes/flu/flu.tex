%St Andrews Notes Template
\documentclass[10pt, a4paper, twocolumn]{article}

%Formatting Packages
\usepackage[a4paper, margin=0.5in]{geometry}
%\usepackage[extreme]{savetrees}
\usepackage{times}

%Math Packages
\usepackage{xparse}
\usepackage{amsmath}
\usepackage{amssymb}
\usepackage{esint}
\usepackage{physics}
\usepackage{cancel}

\newcommand{\deff}[1]{\par \noindent \textit{#1}: }
\newcommand{\dbar}{\mathrm d \hspace*{-0.2em}\bar{}\hspace*{0.2em}}
\newcommand{\pfs}{\ensuremath{\varepsilon _0}}
\newcommand{\arr}{\ensuremath{\longrightarrow\ }}
\newcommand{\larr}{\ensuremath{\longleftarrow\ }}
\newcommand{\intall}{\ensuremath{\int\limits_\text{all space}}}
\newcommand{\intinf}{\ensuremath{\int\limits_{-\infty}^{+\infty}}}
\newcommand{\eps}{\ensuremath{\varepsilon _0}}
\newcommand{\n}{\par \noindent}

\author{Jack Symonds}
\title{Fluids}
\date{}

\begin{document}
\maketitle
\section{introduction}
\subsection{continuum assumption}

\deff{fluid element}
a patch over which we define local variables
\begin{enumerate}

\item A fluid element $L_{el}$ is small enough that we can ignore systematic varations across it:
\[L_{el}\ll L_\mathrm{scale}
\approx\frac{q}{\abs{\grad q}}\]

$q$ is any quantity and $L_{el}$ is a length scale over which $q$ varies by order unity.

\item A fluid element is large enough that it contains enough particles that you can ignore fluctuations due to a finite number of particles.
\[nL_{el}^3\gg 1\]
$n$ is the number density (m$^-3$).

\item A fluid element is large enough that consituent particles "know" about local conditions through colliding with each other:
\[L_{el}\gg\lambda\]
$\lambda$ is the mean free path.
\end{enumerate}

$\therefore$ all quantities are constant throughout a fluid element.
(temp, pressure, density)

\section{relating Eulerian and Lagrangian descriptions}

\[\dv{Q}{t}
=\frac{Q(t+\delta t)-Q(t)}{\delta t}
\qq{\arr}
\dv{Q(r,t)}{t}
=\frac{Q(\vb r+\delta \vb r,t+\delta t
-Q(\vb r,t)}{\delta t}
\]

\begin{equation*}
\begin{aligned}
Q(\vb r+\delta \vb r,t+\delta t)
-Q(\vb r,t)
&=
\underbrace{Q(\vb r,t+\delta t)
-Q(\vb r,t)}
_\text{variation in $t$ at fixed $\vb r$}
+
\underbrace{Q(\vb r+\delta \vb r,t+\delta t)
-Q(\vb r,t+\delta t)}
_\text{variation in $\vb r$ at fixed $t+\delta t$}
\\&=\delta t\pdv{Q}{t}+\ldots
+\delta \vb r\cdot \grad Q +\ldots
\\&\approx
\eval{\pdv{Q}{t}}_t
\delta t+\delta\vb r\cdot
\eval{\grad{Q}}_{t+\delta t}
\\&\approx
\eval{\pdv{Q}{t}}_t
\delta t+\delta\vb r\cdot
\bqty{\grad Q +\delta t\pdv{t} \grad Q}_t
\end{aligned}
\end{equation*}

\[\therefore
\dv{Q}{t}\approx
\pdv{Q}{t}+\frac{\delta\vb r}{\delta t}
\cdot \grad Q\]

\deff{streamline}
a curve that has $\vb u$ in the tangential direction: where mass element travels to in a steady flow ($\pdv{t}=0$)

\begin{equation*}
\begin{aligned}
\frac{\dd x}{u_x}=\frac{\dd y}{u_y}
&\Rightarrow
u_y\dd x-u_x \dd y=0
\\&\Rightarrow
\pdv{\psi}{x}\dd x+
\pdv{\psi}{y}\dd y = \dd \psi =0
\end{aligned}
\end{equation*}

Hence the stream function $\psi$ is constant on a streamline, but differs for different stream lines.

\section{flows through surfaces}
\subsection{Gauss' theorem}

For a box of volume $\Delta x\Delta y\Delta z$, the $x$-component of $\vb u$ at the center of the 'front' and 'back' face is:
\begin{equation*}
\begin{aligned}
u_b&\approx u_x-\frac{\Delta x}{2}\pdv{u_x}{x}
\\u_f&\approx u_x+\frac{\Delta x}{2}\pdv{u_x}{x}
\end{aligned}
\end{equation*}

\[\text{volume crossing back face per time}
=\underbrace{\pqty{u_x-\frac12\pdv{u_x}{x}\Delta x}}
_\text{distance moved/sec}
\underbrace{\Delta y\Delta z}
_\text{area of face}\]

How much fluid is transported through a surface area $A$ is the flux through the surface $A=\Delta y\Delta z$.

\[\text{volume crossing front face per time}
=\underbrace{\pqty{u_x+\frac12\pdv{u_x}{x}\Delta x}}
_\text{distance moved/sec}
\underbrace{\Delta y\Delta z}
_\text{area of face}\]

\[\therefore\text{net vol/sec flowing in $x$-direction}
=\pdv{u_x}{x}\Delta x\Delta y\Delta z\]

\[\therefore\text{total net vol/sec}
=(\div \vb u)\Delta x\Delta y\Delta z\]

\[\text{total flux}
=\iiint_V \div \vb u \ \dd V
\]

\[\text{volume that flows through $\dd S$ each second}
=(\vb u\cdot \vb n)\dd S\]

\[\text{total flux}
=\oiint _S \vb u\cdot \vb{\dd S}
\qquad \dd \vb S = \vb n\ \dd S
\]

In the absence of sources or sinks of mass,
the total rate at which mass (density $\rho$) flows through the surface $S$:
\begin{equation*}
\begin{aligned}
-\sum _i\rho\vb u\cdot \dd \vb S_i
&=-\int _S\rho\vb u\cdot \dd \vb S
\\&=-\int _V\div{(\rho\vb u)} \dd V &
\\&&=\pdv{t}\int _V \rho\dd V
\end{aligned}
\end{equation*}

\[\therefore
\int _V\pqty{\pdv{\rho}{t}+\div{(\rho \vb u)}}\dd V=0
\]

\[\text{equation of continuity:}\qquad
\pdv{\rho}{t}+\div{(\rho\vb u)}=0\]

\[\text{mass conservation in co-moving frame:}\qquad
\dv{\rho}{t}=-\rho\div\vb u\]

\section{the momentum equation}

For any surface within a fluid there is a momentum flux across it (from each side)
\emph{that has nothing to do with any bulk flow} but is a consequence of its thermal properties.

Microscopically, assume a perfect gas. Meaning that the finite temperature imparts molecules with random motions.
The pressure is the associated (one-sided) momentum flux.

Thermal motion is isotropic, hence, the \underline{local momentum} flux is
independent of the orientation of the surface and always perpendicular to the surface (the parallel components cancel out)

For a lump of fluid subject to gravity and the inward pressure of the surrounding fluid,
pressure force on $\dd\vb S$ is $-p\vb S$
\[\text{component of inward pressure force:}\qquad
-p\vb{\hat n}\cdot \dd \vb S\]

\[-\oiint _Sp\vb{\hat n}\cdot \dd\vb S
=-\iiint _V \div{p\vb{\hat n}}\ \dd V\]

\[\text{total momentum in volume $V$:}\qquad
\smallint _V\rho\vb u\ \dd V\]
\[\text{rate of change of momentum in $V$:}\qquad
\dv{t}\int_V\rho\vb u\ \dd V\]

The equation of motion in direction $\vb{\hat n}$ is the rate of change of momentum equated to the sum of forces:
\[\therefore
\pqty{\dv{t}\int_V\rho\vb u\ \dd V}
\cdot \vb{\hat n}
=-\int _V\div{(p\vb{\hat n})}\ \dd V
+\int _V\rho\vb g\cdot\vb{\hat n}\ \dd V
\]

\[\div{(p\vb{\hat n})}=\vb{\hat n}\cdot\grad p
+\cancelto{0}{p\div{\vb{\hat n}}}\]

assuming lump is small,
\[\dv{t}(\rho\vb u\delta V)\cdot\vb{\hat n}
=\cancelto{0}{\vb u\cdot\vb{\hat n}
\tfrac{\dd}{\dd t}(\rho\delta V)}
+\rho\delta V\dv{\vb u}{t}\cdot \vb{\hat n}\]

\begin{equation*}
\begin{aligned}
\therefore
\rho\delta V\dv{\vb u}{t}\cdot\vb{\hat n}
&=\delta V(-\grad p +\rho\vb g)\cdot \vb{\hat n}
\\ \delta V\pqty{\rho\dv{\vb u}{t}+\grad p-\rho\vb g}
\cdot\vb{\hat n} &=0
\end{aligned}
\end{equation*}

\deff{equation of motion,momentum/conservation equation (Lagrangian form)}
\[\rho\dv{\vb u}{t}=-\grad p+\rho\vb g\]

\deff{Eulerian form}
\[\rho\pdv{\vb u}{t}+\rho(\vb u\cdot
\grad{})\vb u
=\grad p+\rho \vb g\]
The momentum contained in a fixed grid cell changes in response to external forces (pressure and gravitational forces) plus any imbalance in the momentum flux in and out of the cell.

The \emph{thermal pressure} is associated with random motions in the fluid which are isotropic, and is scalar (acts the same way in any direction).

The \emph{ram pressure} is associated with bulk motion of the fluid which is oriented. Only a surface whose normal has some component along the direction of flow feels the ram pressure.

\deff{$2^\text{nd}$ law of thermodynamics:}
\[T\dd S=\dd Q
\dd U+p\dd V\]
\[
\mqty{
\dd Q=\delta m\ \dd q
\\ \dd U=\delta m\ \dd e
\\ \dd V=\delta m\ \dd \pqty{\frac1p}
}\Bigg\}\arr
T\dd s=\dd q=\dd e+p\ \dd \pqty{\frac1p}
\]

\[\dd s=\frac{\dd S}{\delta m}
\qquad
e=\frac{p}{(\gamma-1)\rho}
\qquad
\gamma=\frac{c_p}{c_v}\]

$L$ is the sum of sources and sinks of energy, we can differentiate by $\dd t$ (and multiply by density) to get:
\[\rho T\dv{s}{t}
=\rho\bqty{\dv{e}{t}
+p\dv{t}\pqty{\frac1p}}=-L\]

\begin{equation*}
\begin{aligned}
\rho T\dv{s}{t}
&=\rho\bqty{\dv{e}{t}
+p\dv{t}\pqty{\frac1p}}=-L
\\&=\rho\dv{e}{t}+p\rho
\pqty{\frac{-1}{p^2}}\dv{\rho}{t}
\\&=\rho\dv{e}{t}
-\frac{p}{\rho}\dv{\rho}{t}
\qquad \text{(apply mass conservation)}
\\&=\rho\dv{e}{t}+p\div{\vb u}=-L
\end{aligned}
\end{equation*}

\begin{equation*}
\begin{aligned}
\vb u\cdot \bigg(\rho\dv{\vb u}{t}
&=-\grad{p}+\rho\vb g\bigg)
\\ \rho\dv{t}\pqty{\tfrac12 u^2}
&=-u\cdot\grad p+\vb u\cdot \rho\vb g
\end{aligned}
\end{equation*}
\begin{equation*}
\begin{aligned}
\rho\dv{t}\pqty{\frac12 u^2+e}
&=-L-p\div\vb u-\vb u\cdot\grad p
+\vb u\cdot \rho \vb g
\\&=-L-\div p\vb u
+\vb u \cdot\rho\vb g
\end{aligned}
\end{equation*}

\begin{equation*}
\begin{aligned}
\rho\dv{A}{T}
&=\dv{(\rho A)}{t}-A\dv{\rho}{t}
\qquad\text{(apply mass conservation)}
\\&=\dv{(\rho A)}{\dd t}
+\rho A\div{\vb u}
\qquad\text{(convert into observers frame)}
\\&=\pdv{(\rho a)}{t}
+\vb u\cdot \grad{(\rho A)}
+\rho A\div{\vb u}
\\&=\pdv{(\rho A)}{t}+\div{(\rho A)}\vb u
\end{aligned}
\end{equation*}

\[\pdv{t}
\pqty{\frac12 \rho u^2+\rho e}
+\div{
\pqty{\frac12 \rho u^2+\rho e}
}\vb u
=-L-\div{p\vb u}-\vb u\cdot \rho\grad \psi
\]

\begin{equation*}
\begin{aligned}
\div{(\rho \psi)\vb u}
&=(\rho \psi )\div \vb u
+\vb u\cdot\grad{(\rho \psi)}
\\&=(\rho \psi)\div \vb u
+\psi \vb u\cdot \grad\rho
+\rho \vb u\cdot \grad \psi
\\&=\psi(\rho \div \vb u+\vb u\cdot \grad\rho)
+\rho \vb u \cdot\grad\psi
\\ \text{(use mass conservation)}
&=\psi\pqty{-\pdv{\rho}{t}}
+\rho \vb u\cdot\grad\psi
\end{aligned}
\end{equation*}

\section{equations of state}

\[\div\pqty{\frac12\rho u^2
+pe+p+\rho \psi}\vb u=-L
\qquad\textit{enthalpy: }
\rho e+p=\frac{\gamma}{\gamma-1}p
\]

In steady state, the net effect $L$ of the sources and sinks of energy is equal to the flux of energy through the surface of the volume.

In general, $p=p(\rho,T)$ and for an ideal gas:
\[p=nk_BT
\qq{or}
p=\frac{k_B}{m}\rho T\]

\deff{barotropic equation of state}
$p$ is a function of $\rho$ only.

To approximate a fluid as being isothermal
($T=\text{constant}
\longrightarrow p\propto\rho$)
we require that
\begin{itemize}
\item temperature for thermal equilibrium isn't very sensitive to the heating/cooling rate
\item in time-dependent problems, therei s time for the system to reach this constant $T$ thermal equilibrium
\end{itemize}

\deff{adiabatic equation of state}
\[p=K\rho ^\gamma
\qquad \gamma = \frac{c_p}{c_b}\]

This is derived from the ideal gas laws assuming no heat excange with surroudings (only $p\dd B$ work).

A fluid element behaves adiabatically if $K$ is constant as the element's properties change.

An isentropic fluid is one in which all the elements have the same value of $K$.

\begin{equation*}
\begin{aligned}
-L&=\rho\dv{e}{t}-\frac{p}{\rho}\dv{\rho}{t}
\qquad e=\frac{p}{(\gamma-1)\rho}
\\& =\frac{\rho}{\gamma -1}
\dv{t}\pqty{\frac{p}{\rho}}
-\frac{p}{\rho}\dv{\rho}{t}
\\&=\frac{1}{\gamma -1}
\pqty{\dv{p}{t}-\frac{p}{\rho}\dv{\rho}{t}
-\frac{(\gamma -1)p}{\rho}\dv{\rho}{t}}
\\&=\frac{1}{\gamma -1}
\pqty{\dv{p}{t}-\frac{\gamma p}{\rho}
\dv{\rho}{t}}
\\&= \frac{\rho ^\gamma}{\gamma -1}
\dv{t}\pqty{\frac{p}{\rho ^\gamma}}
\end{aligned}
\end{equation*}

\deff{vorticity}
tendency for parcel of fluid to rotate about an axis through its centre of mass
\[\vb \omega =\curl \vb u\]

\[\mathrm{curl} \vb F =
\begin{vmatrix}
\vb i&\vb j&\vb k
\\ \pdv{x}&\pdv{y}&\pdv{z}
\\ F_x&F_y&F_z
\end{vmatrix} \]

\deff{rigid body rotation}
reach parcel of fluid changes its orienation as it moves (as opposed to circulation without rotation)

\deff{viscosity}
the internal stress (force/unit area) from a fluid dragging other fluid

stress: (for a Newtonian fluid)
\[\tau = \mu \pdv{u}{y}\]
$\tau$ is the coefficient of shear viscosity, and $\tau$ is the stress tensor. For a given flow $\vb u$, thte higher the viscosity, the greater the stress.

viscous force/element of volume (stress):
\[
\bqty{
  \mu \eval{\pdv{u}{y}}_{y+\delta y}
  -\mu \eval{\pdv{u}{y}}_{y}
}
\delta x\ \delta z
=\bqty{
  \pdv{y}\pqty{\mu\pdv{u}{y}}
  \delta y
}
\delta x\ \delta z
=\mu\pdv[2]{u}{y}\delta y\ \delta x\ \delta z
\]

\deff{viscous stress force}
\[ F_\nu=\mu\pdv[2]{u}{y}\]

generalizing:
\[F_\nu = \rho\nu \grad^2{u}
\qquad \mu = \rho\nu \]

new equation of motion:
\[\pdv{\vb u}{t}
+(\vb u\cdot \grad{})\vb u
=-\frac{1}{\rho}
\grad p
+\vb g + \nu\grad ^2\vb u
\]







\end{document}
